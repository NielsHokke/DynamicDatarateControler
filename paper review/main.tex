\documentclass [a4,twoside,11pt] {article}

\usepackage{times}

\setlength{\textwidth}{6in}
\setlength{\textheight}{10in}
\setlength{\oddsidemargin}{0.25in}
\setlength{\evensidemargin}{0.25in}

\hyphenation{practicum-ad-mini-stratie}

\begin{document}

\title{
   \center {
      \LARGE{PAPER REVIEW \\ WIRELESS NETWORKING (ET4394) \\ 2017-2018}
   }
}
\author{Jetse Brouwer, 4615964 \\ Niels Hokke, 4610148}
\date{April 3th, 2018}
\maketitle

\setcounter{secnumdepth}{1}

% ============================================================================

\section{Paper summarized}
Pengyu Zhang, Mohammad Rostami, Pan Hu, Deepak Ganesan ``Enabling Practical Backscatter Communication for On-body Sensors'' SIGCOMM ’16, August 22-26, 2016, Florianopolis , Brazil


\section{Summary}
The paper suggests a way to implement Wi-Fi and Bluetooth back-scatter to be used for a on-body sensor that is compatible with customer hardware (Hardware that does not support OFDM carrier cancellation). One of the uses cases mentioned is a situation where a phone would broadcast a carrier signal, a on-body tag would back-scatter a signal which then could be received by a smart watch or fitbit-like device.

As smart-phones or access-points don't support carrier cancellation the ratio between the carrier and the back-scatter signal would be to low to have a feasible communication. To overcome this the researchers propose a way of frequency shifting the back-scattered signal by 20 Mhz (as this is a single channel shift for Wi-Fi or a 10 channel shift for Bluetooth) to an empty channel. One of the novelties presented in the paper is the extremely low power modulator used for the frequency shift. This is implemented with the use of a ring-modulator build from not-gates. The modulator is very sensitive for temperature fluctuations, however as this is designed for on-body sensing the temperature will not vary much more then 2 degrees and the frequency will remain within the range of th receiver. The power consumption of the designed modulator is 20 $\mu$W compared to 2 mW for commercially available oscillators. 

For transmitting data they suggest two methods; one is the packet-level decoding where a packet shifted represents a logical 1 and a packet not shifted represents a logical 0. This exploits the fact that Wi-Fi and Bluetooth receivers are designed to be extremely sensitive to structured weak signals, such as the preamble in a packet. (e.g the modem used has a sensitivity of -95 dBm) With this method they achieved a range of 4.8 meter and an average throughput of 628 bits per second. The second method is bit-level decoding, here the signal is shifted or not shifted per bit of the packet. This results in a higher throughput, but we lose the leverage of the high sensitivity of the low signal packets and therefor reducing the range to 3.5 meter but increasing the throughput up to 50 Kbs.

\newpage{}
\section{Assessment}

The paper is well written and clearly structured which makes for an easy read. Every subsection starts with stating a question which is then answered, this makes it very clear which problem they are trying to solve. When suggesting a new design it's important to make the design verifiable, and to do so the setup must be well defined. The paper clearly states which hardware is used, but does not provide the software. This isn't a large problem as the novelty of the new method is mostly hardware based. The software described uses the Wifi/Bluetooth radio's API's, so documentation should be available online.

One possibly problematic assumptions they make is that the adjacent channel is empty. They offer a simply solution such as using cts-rts frames from the initiator to ensure the tag is channel is empty. However as the channel is then only free for a limited amount of time, this implies that either the tag should be able to decode the frames, making it computationally heavy, or lowering the amount of time per transmission, introducing more overhead. 
Also if one wants to transmit it must ensure that both the Carrier and the back-scatter channel is free which might be a problem for crowded Wi-Fi spectrum.

\section{Potential improvement(s)}
The paper talks about a low power solution but does not talk about an actual passive system including energy harvesting. As the paper only suggest a new way to communicate this isn't an large improvement, but it would be interesting to see the capabilities of the whole packages as a passive wireless sensor.

Another improvement would be if the authors also published the used software, this would make it easier to get the basic setup running and to make further improvements.

\end{document}
